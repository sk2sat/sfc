\documentclass[a4j,10pt]{jsarticle}
\usepackage{layout,url,resume}
\usepackage[dvipdfmx]{graphicx}
\usepackage[dvipdfmx]{hyperref}
\pagestyle{empty}

\begin{document}
%\layout

\title{小型ハイブリッドロケット向けフライトシミュレータの開発}

% 和文著者名
\author{
	Arch B1 坂本優太(sksat)
}

% 和文概要
\begin{abstract}
小型ハイブリッドロケットを対象としたフライトシミュレータを新規に開発した.
\end{abstract}

\maketitle
\thispagestyle{empty}

\section{開発の経緯}
% COREとかいうのに入ってる
% ハイブリッドロケット is 何
% 安全審査 is 何

筆者は"CORE"\footnote{"Challengers Of Rocket Engineering"の頭文字をとったもの.}
という学生ロケット
\footnote{「学生ロケット」という言葉には明確な定義は無いが,ここでは学生を主体とした小型のロケットの開発・打ち上げ活動とする.}
サークルに所属している.
COREでは主に小型ハイブリッドロケットの開発・打ち上げ実験を行っている.

これらのロケットを設計する際,
ロケットがどのように・どの程度飛翔すかといった検討を行う.
ロケットの飛翔を解析的に解くのは難しい
\footnote{ロケットはその性質上,質量・推力を刻一刻と変化させながら飛翔する上に,
空力による影響が大きい.解析的に解くのはとても大雑把な検討をする際のみであろう.}
ため,コンピュータを使って数値計算(以後シミュレーションと呼称)を行う必要がある.
また,このシミュレーションはロケットを打ち上げるにあたって事前・直前に実施される安全審査においても要求されるものとなっており,
打上における安全確保の観点からも重要なものである.

このシミュレーションを行うために,
COREでは過去にOBが開発したシミュレータ"FROGS"
\footnote{Flight Of ROcket and Ground-hit-points Simulator.}
を使用してきた.
しかし,筆者がシミュレーション担当となり,
8,11月の打上実験においてFROGSを使用したところ,様々な問題点があることが分かった.
これらの問題点を解消するため,今回新たなフライトシミュレータを開発するに至った.

% ハイブリッドロケットは液体燃料ロケット・固体燃料ロケットと並ぶ化学反応を利用した種類のロケット推進システムの1つである.
% ハイブリッドロケットが「ハイブリッド」と呼ばれる所以は,
% 相の異なる2種類の推進剤を用いるという特徴にある.
% 最も多いのは液体の酸化剤と固体の燃料を用いる方式であり,
% これはいわば液体燃料ロケットと固体燃料ロケットを足して2で割ったようなものだ.
% 実際に,ハイブリッドロケットは双方のロケットシステムの利点を兼ね備えている.
% 
% ハイブリッドロケットは本質的に爆発しないと言われており,
% その他のロケットに比べ安全性が高い.
% そのため,大学生でも扱うことができるよん.

\section{既存のシミュレータの問題点}

% FROGSが如何にカスか
% matlabだとインカレで困る
% コードがカス: 関数なにそれおいしいの
% パラメータ入力がカス
% 可視化もカス

まず重要なこととして,FROGSがMatlab向けに書かれたプログラムだということが挙げられる.
COREはインターカレッジ・サークルであり,首都大学東京
\footnote{2020年4月以降"東京都立大学"に改称予定.}
を中心とした様々な大学の学生が集まった団体となっている.
なので,中にはMatlabのライセンスを学生に配布していない大学の学生もいる.
Matlabのライセンスは一介の大学生に購入できるものではないため,
必然的にCORE内でFROGSを使うことができる人間は限られてしまう.

また,FROGSはあまりソフトウェア開発の経験が無い人によって開発・保守されてきた.
これにより,コードの再利用性が著しく低い上,
Google Drive上でフォルダ名によってバージョン管理がなされている,
といった運用状況があった.
これは個人的に許容できるものではなかったため,
文字コードの変更やインデントの修正を行ったものをGitに取り込み,
https://github.com/core-rocket/FROGS
\footnote{そもそも,https://github.com/core-rocket というorganizationも筆者が作成・整備したものだ.}
にて公開した.
2019年11月の伊豆大島共同打上実験と,
現在プロジェクトが進行中の2020年3月の伊豆大島共同打上実験において
筆者が担当するシミュレーションはこのリポジトリで行っている.

さらに,計算結果の可視化に関しても,FROGSには問題があった.
まず,計算結果のファイル保存を行わないため,
可視化を行うためには毎回計算を実行しなければならなかった.
1ケースのみの可視化をしたい場合であれば,
計算時間も短いのでこれは大した問題にはならない.
だが,次に述べる落下分散図の作成時や,再計算を要求された時など,
複数のケースを検討する必要がある場合,
FROGSではこれは根気の要る作業となる.

落下分散図は,小型ハイブリッドロケットでは(特に安全審査において)最も重要な可視化である.
これはは,ある条件
\footnote{ランチャの仰角が何度か,パラシュートが正常に開いた場合や開かずロケットが弾道してしまった場合,といったものが実際に検討する条件となる.}
でロケットを発射した際に,
ロケットが落下する地点の座標を複数の風向・風速毎に,
図\ref{ghp-example}のようにプロットしたものだ.

\begin{figure}[htbp]
	\begin{center}
		\includegraphics[width=7cm]{./ghp-example.png}
		\caption{落下分散図の例}
		\label{ghp-example}
	\end{center}
\end{figure}

どのようなロケットにおいても,
それが上空から落下してくる可能性がある以上,
「ここに落としてはいけない」という落下禁止区域が存在する.
落下禁止区域は土地の所有者や漁協との兼ね合いに加え,
打上関係者を含む全ての人間の安全のために設定される.


\section{先行事例}
% FROGS, ForRocket, OpenTsiolokvsky
% OpenRocketは先行事例 && 先行研究みがある
% OpenRocketはプロトタイプには良いが

COREのFROGS以外にも,多くの学生ロケット団体が独自にシミュレータを開発している.
しかし,学生ロケットの分野においてはまだオープンソースの文化が浸透していない.
そのため,挙げられる先行事例は多くない.

ForRocket

また,ハイブリッドロケットではなく,
モデルロケットという小型の固体燃料ロケットを対象にしたシミュレータも存在する.
このようなシミュレータのうち,オープンソースで代表的なものはOpenRocketである.
\footnote{シミュレータというよりは設計とシミュレーションを行うための統合環境なのだが,
シミュレータ部分で行っていることはハイブリッドロケットのシミュレーションと似たようなものであるため,先行事例とした.}
小さなモデルロケットの設計・シミュレーションには申し分ないソフトウェアとなっており,
小型ハイブリッドロケットの設計・シミュレーションも可能ではあるため,
小型ハイブリッドロケット開発ではプロトタイプの検討に使われることが多い.
だが,圧力中心の計算がポテンシャル流れを仮定したものである他,
フィンに台形以外の形状を使用した際に著しく正確性を欠くなどの問題点があり,
伊豆大島共同打ち上げ実験・能代宇宙イベントでは推奨されていない.
元はヘルシンキ大学の修士論文
\footnote{Development of an Open Source model rocket simulation
software}
のためのプロジェクトであり,先行研究とも言える.


\section{実装}

\subsection{シミュレータ本体}

実装にはC++17を使用した.
\footnote{今後C++20に移行する可能性がある.}

\subsection{苦労した点}
% MinGWがC++17にちゃんと対応してなくてカス
% Windowsの浮動小数点演算精度が違う
% なんかミスって全部NaN


\section{評価}
\subsection{定性的評価}
% わあい使いやすいね

\subsection{定量的評価}
% 高速化
% 近いところに落下分散
% → 頂点到達後しばらくしてから意図しない回転があり,空力まわりにバグ?

\section{今後の課題}
% バグ修正
% 風向風速制限表の自動生成(できたけど,不十分なのでもうちょいやる)
% FROGS・3月大島データと定量的に比較して妥当性確認
% 燃焼試験データオープンデータ化
% 	既存フォーマットだと不十分,テキストだとサイズがデカい→独自フォーマット考案中
% ペイロード・多段式ロケット実装
% 高度11km以上上空の大気モデル
% マッハ0.8対応
% 精度保証付き数値計算?
% パラメータ最適化

\section{まとめ}

% \bibliographystyle{junsrt}
% \bibliography{resume}

\begin{thebibliography}{99}
	\bibitem{openrocket-thesis}
		Sampo Niskanen. ; Development of an Open Source model rocket simulation software. (2013-05-20)
	\bibitem{rocket-system}
		田辺英二.ロケットシステム(1999)
	\bibitem{rocket-propulsion-elements}
		George P. Sutton. ; ロケット推進工学.望月晶訳.(1992)
	\bibitem{hybrid-rocket-know-how}
		import Avio.小型ハイブリッドロケットノウハウ vol.1.(2019)
	\bibitem{numerical-calc-common-sense}
		伊里正男,藤野和建.数値計算の常識.(1985)
\end{thebibliography}

\end{document}
% end of file
