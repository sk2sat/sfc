\documentclass[a4j,10pt]{jsarticle}
\usepackage{layout,url,resume}
\usepackage[dvipdfmx]{graphicx}
\usepackage[dvipdfmx]{hyperref}
\pagestyle{empty}

\begin{document}
%\layout

\title{小型ハイブリッドロケット向けフライトシミュレータの開発}

% 和文著者名
\author{
	Arch B1 坂本優太(sksat)
}

% 和文概要
\begin{abstract}
うんち
\end{abstract}

\maketitle
\thispagestyle{empty}

\section{背景}
% COREとかいうのに入ってる
% ハイブリッドロケット is 何
% 安全審査 is 何

\section{既存のシミュレータの問題点}

% FROGSが如何にカスか
% matlabだとインカレで困る
% コードがカス: インデントなにそれおいしいの,関数ry
% パラメータ入力がカス
% 可視化もカス

\section{先行事例}
% FROGS, ForRocket, OpenTsiolokvsky
% OpenRocketは先行事例 && 先行研究みがある
% OpenRocketはプロトタイプには良いが

\section{実装}

\subsection{苦労した点}
% MinGWがC++17にちゃんと対応してなくてカス
% Windowsの浮動小数点演算精度が違う
% なんかミスって全部NaN

\section{評価}
\subsection{定性的評価}
% わあい使いやすいね

\subsection{定量的評価}
% 高速化
% 近いところに落下分散
% → 頂点到達後しばらくしてから意図しない回転があり,空力まわりにバグ?

\section{今後の課題}
% バグ修正
% 風向風速制限表の自動生成(できたけど,不十分なのでもうちょいやる)
% FROGS・3月大島データと定量的に比較して妥当性確認
% 燃焼試験データオープンデータ化
% 	既存フォーマットだと不十分,テキストだとサイズがデカい→独自フォーマット考案中
% ペイロード・多段式ロケット実装
% 高度11km以上上空の大気モデル
% マッハ0.8対応
% 精度保証付き数値計算?
% パラメータ最適化

\section{まとめ}

% \bibliographystyle{junsrt}
% \bibliography{resume}

\begin{thebibliography}{99}
	\bibitem{rocket-system}
		田辺英二.ロケットシステム(1999)
	\bibitem{rocket-propulsion-elements}
		George P. Sutton. ; ロケット推進工学.望月晶訳.(1992)
	\bibitem{hybrid-rocket-know-how}
		import Avio.小型ハイブリッドロケットノウハウ vol.1.(2019)
	\bibitem{numerical-calc-common-sense}
		伊里正男,藤野和建.数値計算の常識.(1985)
\end{thebibliography}

\end{document}
% end of file
